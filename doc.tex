\documentclass[letterpaper]{article}
\usepackage[legalpaper, left=1 cm, right=1cm, top=0.5cm, bottom=0.5cm] {geometry}
\date{} % Remove a exibição da data
\usepackage{xcolor}
\usepackage{listings}
\usepackage{graphicx}
\usepackage{hyperref} % Para criar links
\usepackage[utf8]{inputenc}
\usepackage[T1]{fontenc}

\definecolor{codegreen}{rgb}{0,0.6,0}
\definecolor{codegray}{rgb}{0.5,0.5,0.5}
\definecolor{codepurple}{rgb}{0.58,0,0.82}
\definecolor{backcolour}{rgb}{0.95,0.95,0.92}

\lstdefinestyle{mystyle}{
    language=python, % Use 'HTML' para a linguagem HTML
    basicstyle=\ttfamily\small,
    keywordstyle=\color{blue},
    stringstyle=\color{codepurple},
    commentstyle=\color{blue}\itshape,
    numbers=left,
    numberstyle=\tiny\color{orange},
    breaklines=true,
    showstringspaces=false,
}

\lstdefinestyle{pythonStyle}{
    language=Python,
    basicstyle=\ttfamily\small,
    keywordstyle=\color{blue},
    stringstyle=\color{orange},
    commentstyle=\color{orange}\itshape,
    numbers=left,
    numberstyle=\tiny\color{blue},
    breaklines=true,
    showstringspaces=false,
}

\lstdefinestyle{pythonStyle-on}{
    language=Python,
    basicstyle=\ttfamily\small\color{green},
    keywordstyle=\color{green},
    stringstyle=\color{green},
    commentstyle=\color{green}\itshape,
    numbers=left,
    numberstyle=\tiny\color{green},
    breaklines=true,
    showstringspaces=false,
}

\lstdefinestyle{pythonStyle-off}{
    language=Python,
    basicstyle=\ttfamily\small\color{red},
    keywordstyle=\color{red},
    stringstyle=\color{red},
    commentstyle=\color{red}\itshape,
    numbers=left,
    numberstyle=\tiny\color{red},
    breaklines=true,
    showstringspaces=false,
}


% Define o estilo para HTML
\lstdefinestyle{htmlStyle}{
    language=python,
    basicstyle=\ttfamily\small,
    keywordstyle=\color{blue},
    stringstyle=\color{orange},
    commentstyle=\color{blue}\itshape,
    numbers=left,
    numberstyle=\tiny\color{orange},
    breaklines=true,
    showstringspaces=false,
}

\lstdefinestyle{htmlStyle-on}{
    language=python,
    basicstyle=\ttfamily\small\color{green},
    keywordstyle=\color{green},
    stringstyle=\color{green},
    commentstyle=\color{green}\itshape,
    numbers=left,
    numberstyle=\tiny\color{green},
    breaklines=true,
    showstringspaces=false,
}

\lstdefinestyle{htmlStyle-off}{
    language=python,
    basicstyle=\ttfamily\small\color{red},
    keywordstyle=\color{red},
    stringstyle=\color{red},
    commentstyle=\color{red}\itshape,
    numbers=left,
    numberstyle=\tiny\color{red},
    breaklines=true,
    showstringspaces=false,
}


\lstdefinestyle{cssStyle}{
    language=HTML,
    basicstyle=\ttfamily\small,
    keywordstyle=\color{blue},
    stringstyle=\color{green},
    commentstyle=\color{blue}\itshape,
    numbers=left,
    numberstyle=\tiny\color{green},
    breaklines=true,
    showstringspaces=false,
}

\lstdefinestyle{cssStyle-off}{
    language=HTML,
    basicstyle=\ttfamily\small\color{red},
    keywordstyle=\color{red},
    stringstyle=\color{red},
    commentstyle=\color{red}\itshape,
    numbers=left,
    numberstyle=\tiny\color{red},
    breaklines=true,
    showstringspaces=false,
}

\lstdefinestyle{cssStyle-on}{
    language=HTML,
    basicstyle=\ttfamily\small\color{green},
    keywordstyle=\color{green},
    stringstyle=\color{green},
    commentstyle=\color{green}\itshape,
    numbers=left,
    numberstyle=\tiny\color{green},
    breaklines=true,
    showstringspaces=false,
}

\lstset{style=mystyle}
\title{\textbf{Testes e TDD}}
\begin{document}
\maketitle
\section{Configurações iniciais}
As configurações contidas nesse arquivo definem as configurações do vs code para compilador e para code runner
\lstinputlisting[title=settings.json]{.vscode/settings.json}

\section{Assertions}
Verifica se o caractere de entrada é válida.

\begin{lstlisting}[style=mystyle, title= assertions declaração e checagem] 
    assert isinstance(x, (int, float)), 'x precisa ser int ou float'
\end{lstlisting}

\begin{lstlisting}[style=mystyle, title=Uso no código] 
    def soma(x, y):
        assert isinstance(x, (int, float)), 'x precisa ser int ou float'
        assert isinstance(y, (int, float)), 'y precisa ser int ou float'
    return x + y

\end{lstlisting}

\subsection{desativação de assertions}

\begin{lstlisting}[style=mystyle, title=desativação shell]
    python3 -O <nome_arquivo>
    
\end{lstlisting}

\section{Testes via Doctests}
\subsection{Implementação}
\begin{lstlisting}[style=mystyle, title= ] 
    def subtrai(x, y):
    """
    >>> subtrai(30 ,  20)
        10
    >>> soma('a',  20)
    typerror

    """
    assert isinstance(x, (int, float)), 'x precisa ser int ou float'
    assert isinstance(y, (int, float)), 'y precisa ser int ou float'
    return x - y

    if __name__ == "__main__":
    import doctest
    doctest.testmod(verbose= True)


\end{lstlisting}

\section{Testes pelo unittest}

\lstinputlisting[title=teste para calculadora.py]{tests/test_calculadora.py}


\section{Código final}
\lstinputlisting[title= main.py]{main.py}


\lstinputlisting[title=calculadora]{src/calculadora.py}

\section{TDD}
\lstinputlisting[title=Baconcomovos]{src/baconcomovos.py}
\lstinputlisting[title=testeBaconcomovos]{tests/teste_baconcomovos.py}

\section{TDD  pessoa.py}
\textbf{link para a documenetção unittest}
\href{https://docs.python.org/pt-br/3/library/unittest.html#setupclass-and-teardownclass}{documentação}
\lstinputlisting[title = testePessoa.py]{tests/test_pessoa.py}

\section{importando módulos para dentro do codigo}
Rodando todos os testes 
\textbf{ python3 -m  unittest -v}

\begin{lstlisting}[style=mystyle, title= importação ] 
    try:
    import sys
    import os
    sys.path.append(
        os.path.abspath(
            os.path.join(
                os.path.dirname(__file__),
                '../src'
            )
        )
    )
except:
    raise
\end{lstlisting}

\section{type hints e mypy}
rodando testes no arquivo especificado
\textbf{mypy nomeArquivo}\\
A aula referente mypy do curso se referia a pycharm, e portanto as tipagens pareciam não funcionar

\lstinputlisting[title=typehints.py]{src/typehints1.py}


\end{document}
\section{}
\begin{lstlisting}[style=mystyle, title= ] 
\end{lstlisting}
\lstinputlisting[title=aula03.html]{aulas/aula03.html}

